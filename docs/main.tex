\documentclass{article}

\usepackage[margin=1.25in]{geometry}
\usepackage[utf8]{inputenc}

\usepackage{amsmath}
\usepackage{physics}
\usepackage{url}

\title{Permanents}

\linespread{1.25}

\begin{document}

\section*{Permanents}

The permanent of an $N$-by-$N$ square matrix $A = \left(a_{i,j}\right)$ is defined as
\begin{equation} \label{eq:per1}
    \text{per}(A) = \sum_{\sigma \in P(N)}{\prod_{i=1}^N{a_{i,{\sigma(i)}}}}
\end{equation}
where $P(N)$ is the set of permutations of the $N$-set $\left\{1,\dots,N\right\}$
\cite{wiki:permanent}. Equation (\ref{eq:per1}) is similar to the definition of the determinant,
\begin{equation} \label{eq:det1}
    \text{det}(A) = \sum_{\sigma \in P(N)}{\text{sgn}(\sigma) \prod_{i=1}^N{a_{i,{\sigma(i)}}}},
\end{equation}
except that the signatures of the permutations are not taken into account; this exclusion notably
makes the permanent much more difficult to compute than the determinant, because it means that
decomposition methods can no longer be used \cite{wiki:computing}. In 1979, Valiant published his
theorem stating that the computation of the permanent is in the complexity class \#P-complete, i.e.
neither in P nor in NP \cite{valiant1979}. The permanent, however, commonly appears in problems
related to quantum mechanics, so it is still worthwhile to pursue more efficient algorithms than the
naive ones suggested by the permanent's definition, which has time complexity $\mathcal{O}(N!N)$.

To date, there are two general algorithms considered to be the fastest; one due to Ryser (1963)
\cite{ryser1963}, and one due to Glynn (2010, although it was found independently by several others
prior to this) \cite{wiki:computing,glynn2010}. The Ryser algorithm is based on the
inclusion-exclusion principle and is given by the formula
\begin{equation} \label{eq:per2}
    \text{per}(A) = \left(-1\right)^N \sum_{k=1}^N{
        \sum_{\sigma \in P(N,k)}{
            {\left(-1\right)}^{k}
            \prod_{i=1}^N{
                \sum_{j=1}^{k}{a_{i,{\sigma(j)}}}
            }
        }
    }
\end{equation}
where $P(N,k)$ is the set of $k$-permutations of the $N$-set $\left\{1,\dots,N\right\}$

The Glynn algorithm is based on invariant theory, derived from the polarization identity for
a symmetric tensor, and is given by the formula
\begin{equation} \label{eq:per3}
    \text{per}(A) = \frac{1}{2^{N-1}} \cdot \sum_{\delta}{
        \left(\sum_{k=1}^N{\delta_k}\right)
        \prod_{j=1}^N{\sum_{i=1}^N{\delta_i a_{i,j}}}
    }
\end{equation}
where the outer sum is over all $2^{N-1}$ vectors $\delta = \left\{\pm 1_1,\dots,\pm 1_N\right\}$.
It is still not clear which algorithm is faster, but they both scale in time with $\mathcal{O}(2^N
N^2)$ if implemented naively, and with $\mathcal{O}(2^N N)$ if the sets $P(N,k)$ in Equation
(\ref{eq:per2}) and $\delta$ in Equation (\ref{eq:per3}) are iterated over in Gray code order
\cite{wiki:computing,knuth2005}.

At small values of $N$, the naive algorithm is actually the fastest, but it is quickly outpaced by
the other two algorithms, and the difference in time between the algorithms at large values of $N$
is several orders of magnitude. We haven't been able to find any fast C/++ code for these
algorithms, though, in order to properly determine which algorithm is best at which values of $N$.
It would be nice to have a good C/++ implementation of the permanent function which automatically
deploys the fastest algorithm for the size of the input matrix.

\section*{Permanents of rectangular matrices}

The definition of the permanent can be generalized to work with $M$-by-$N$ rectangular matrices with
$M \leq N$ \cite{wiki:permanent}, giving us the formula
\begin{equation} \label{eq:rectper1}
    \text{per}(A) = \sum_{\sigma \in P(N,M)}{\prod_{i=1}^M{a_{i,{\sigma(i)}}}}.
\end{equation}
Since $\text{per}(A) = \text{per}(A^T)$, this is sufficient for computing
the permanent of all rectangular matrices. Ryser also generalized his algorithm to work with
rectangular permanents in the same 1963 publication \cite{wiki:permanent,ryser1963},
\begin{equation} \label{eq:rectper2}
    \text{per}(A) = \sum_{k=1}^{M}{
        \sum_{\sigma \in P(N,M-k)}{
            {\left(-1\right)}^{k-1} \left(\begin{matrix}N - M - k - 1\\ k - 1\end{matrix}\right)
            \prod_{i=1}^M{
                \sum_{j=1}^{M-k}{a_{i,{\sigma(j)}}}
            }
        }
    }.
\end{equation}

The Glynn algorithm can be generalized to work with rectangular permanents by use of the identity
(shown here for $M \geq N$),
\begin{equation} \label{eq:padidentity}
    {\text{per}}\left(
        \begin{matrix}
            a_{1,1} & \cdots & a_{1,N} \\
            \vdots & \ddots & \vdots \\
            a_{M,1} & \cdots & a_{M,N} \\
        \end{matrix}
    \right)
    = \frac{1}{\left(M - N + 1\right)!} \cdot {\text{per}}\left(
        \begin{matrix}
            a_{1,1} & \cdots & a_{1,N} & 1_{1,N+1} & \cdots & 1_{1,M} \\
            \vdots & \ddots & \vdots & \vdots & \ddots & \vdots \\
            a_{M,1} & \cdots & a_{M,N} & 1_{M,N+1} & \cdots & 1_{M,M} \\
        \end{matrix}
    \right),
\end{equation}
which can be neatly fit into Equation (\ref{eq:per3}) by extending the sum over $\delta$:
\begin{equation} \label{eq:rectper3}
    \text{per}(A) = \frac{1}{2^{N-1}} \cdot \frac{1}{\left(N - M + 1\right)!} \cdot \sum_{\delta}{
        \left(\sum_{k=1}^N{\delta_k}\right)
        \prod_{j=1}^N{\left(\sum_{i=1}^M{\delta_i a_{i,j}} + \sum_{i=M+1}^N{\delta_i}\right)}
    }.
\end{equation}

Now, our problem is more complicated, since we want to determine the best algorithm for each value
of $M$ and $N$. It seems to me that for large rectangular matrices with $M$ and $N$ close in value,
the Glynn algorithm should be fastest; and for large highly rectangular matrices where the
difference between $M$ and $N$ is large, the Ryser algorithm should be fastest.

\section*{Derivatives of Permanents}

We can compute the derivative of a permanent with respect to a general parameter $t$ by using
a formula analogous to Jacobi's formula \cite{carvalho2014},
\begin{equation} \label{eq:jacobis}
    \frac{d}{dt}\text{per}(A(t)) = \text{tr}\left(\text{adj}_{+}(A(t))\frac{dA(t)}{dt}\right).
\end{equation}
The function $\text{adj}_{+}(A)$ in Equation (\ref{eq:jacobis}) is the permanental adjugate, wherein
the $i,j$-minors of $A$ are computed with permanents instead of determinants. We can also compute
the derivative of a permanent with respect to a matrix element $a_{ij}$ by using a special case of
Equation (\ref{eq:jacobis}),
\begin{equation}
    \frac{d}{d{a_{i,j}}}\text{per}(A) = {\text{adj}_{+}(A)}_{j,i}.
\end{equation}

\pagebreak

\bibliography{main}
\bibliographystyle{ieeetr}

\end{document}
